%!TEX root = ../main.tex
\chapter{Grundlagen}
\label{cha:Grundlagen}
Das folgende Kapitel beschäftigt sich mit Grundlagen, derer Kenntnis in der weiteren Arbeit vorausgesetzt wird. Im Vordergrund stehen hierbei die Erläuterung der agilen Arbeitsweise und Scrum, insbesondere die individuellen Anpassungen bei kernpunkt, sowie ein kurzer Überblick über den bisherigen Prototypen und den menschzentrierten Entwicklungsprozess.

\section{Agilität und Scrum}
\label{sec:agilitaet_scrum}
Im Jahr 2001 wurde das agile Manifest von der ''The Agile Alliance'', einem Verbund von 17 Softwareentwicklern und Projektleitern aus dem IT-Umfeld, entwickelt und unterzeichnet \cite{agilemanifestosignatories, agilemanifestohistory}. 
\begin{quote}
''Agile Softwareentwicklung bezeichnet Ansätze im Softwareentwicklungsprozess, die die Transparenz und Flexibilität erhöhen und zu einem schnelleren Einsatz der entwickelten Systeme führen sollen, um so Risiken im Entwicklungsprozess zu minimieren. Die Kernidee besteht darin, Teilprozesse möglichst einfach und somit beweglich (=agil) zu halten.''\cite{definitionagil}.
\end{quote}
Das agile Manifest sollte eine Alternative zu den bisherigen schwergewichtigen und unflexiblen Entwicklungsprozessen, wie beispielsweise dem Wasserfallmodell, aufzeigen, welches auf kurzfristige Ereignisse schnell reagieren und sich so besser individualisieren und anpassen lässt \cite{agilemanifestohistory}. \\ \\
Auf Grundlage dieses Manifestes entstanden verschiedene Frameworks und Arbeitsweisen, darunter Scrum. Das Rahmenwerk Scrum wurde von Ken Schwaber und Jeff Sutherland, zwei Mitgliedern der Agile Alliance, bereits in den frühen 1990er-Jahren erfunden und entwickelt. Es existiert nach eigener Aussage, damit  ''Menschen komplexe adaptive Aufgabenstellungen angehen können'', um ''[...] produktiv und kreativ Produkte mit höchstmöglichem Wert auszuliefern.'' \cite{scrumguide}.
\\ \\
Im Mittelpunkt von Scrum stehen selbstorganisierte Teams. Ein solches Team besteht aus Teammitgliedern, einem ''Product Owner'' und einem ''Scrum Master''. Die Teammitglieder, im Gesamten auch Entwicklungsteam genannt, sind der Kern und dienen der Herstellung des Produktes, in diesem Fall der Entwicklung von Software. Wichtig ist hierbei, dass das Team über alle Fähigkeiten verfügt, um seine Aufgaben lösen zu können. Dementsprechend arbeitet es interdisziplinär in verschiedenen Gewerken. Laut Scrum Guide sollte ein Team aus drei bis neun Mitgliedern bestehen. Die Verantwortlichkeit liegt stets beim gesamten Team und nicht bei einzelnen Personen, auch wenn diese bestimmte Tätigkeitsfelder und Fähigkeiten besitzen.
\\\\
Der Product Owner ist verantwortlich für den Wert der Arbeit der Teammitglieder und den wirtschaftlichen Erfolg. Außerdem ist er allein zuständig für das ''Product Backlog''. Das Product Backlog ist eine Auflistung aller Aufgaben, die nötig sind, um das Produkt herzustellen. Es ist flexibel, neue Aufgaben können hinzukommen, andere können entfernt werden, abhängig von den Anforderungen an das Produkt. Das Backlog ist wird vom Product Owner priorisert, auch ist er zuständig für eine ausreichende Beschreibung er Aufgaben.
 //Aufgaben in Stories, User Story Format, Refinement, Story Points, etc. dann SM, dann eevents und inkremente (sprints)

\section{Analoge und Digitale Scrumboards}
\label{sec:digitale_analoge_boards}
Dies ist ein Beispiel für ein wörtliches Zitat.

HIER MUSS DIE EMPIRIE REIN

\section{Architektur und Prototyp zur Synchronisation von Boards}
\label{sec:architektur_prototyp_sync}

Architektur und Prototyp zur Synchronisation von Boards

\section{Menschzentrierter Entwicklungsprozess nach ISO 9241-210}
\label{sec:menschzentrierte_entwicklung_iso9241}

Hier kommt eine ISO hin, muss zitiert werden (wichtig!)
