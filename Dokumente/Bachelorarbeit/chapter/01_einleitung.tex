%!TEX root = ../main.tex
\chapter{Einleitung}
\label{cha:Einleitung}

Dieses Kapitel beschreibt die Domäne und das Problemfeld, in dem sich diese Arbeit bewegt. Außerdem wird die eigene Motivation, sowie die persönliche Bedeutung erläutert. Theoretische Grundlagen, insbesondere zur Agilität und Scrum, werden in diesem Kapitel vorrausgesetzt, jedoch erst in Kapitel \ref{cha:Grundlagen} erläutert.

\section{Kontext und Motivation der Arbeit}
\label{sec:kontext_motivation}
Das agile Framework Scrum hat in den letzten Jahren an Popularität gewonnen. So wird es laut der Studie ''Status Quo Agile 2016/2017'' der Hochschule Koblenz von 63\% der IT-Firmen -und Agenturen genutzt \cite{statusquoagile}. So auch im Unternehmen ''kernpunkt''. Dort wurde Scrum im November 2017 vorgestellt und das Unternehmen durch eine agile Transformation in die neue Arbeitsweise geführt. Grundlage stellt hier der ''Scrum Guide'' dar \cite{scrumguide}, der jedoch teilweise angepasst wurde, um mit dem Tagesgeschäft zu harmonieren. 
\\ \\
Während der Scrum Guide die Arbeit mit nur einem Kunden beschreibt, wurde der Prozess hier auf eine Multiprojektumgebung mit bis zu 8 verschiedenen Kunden und Projekten pro Team angepasst. Die verschiedenen Kunden müssen auf dem Teamboard getrennt werden, damit Stakeholder nur die für sie relevanten Stories sehen. Um dies umzusetzen, verwenden die verschiedenen Teams bei kernpunkt digitale Scrumboards, die auch für Kunden einsehbar sind. Auch wird durch das Board und die dort enthaltenen Stories die Kommunikation zwischen Kunde und Team bestimmt. In Dailies und weiteren teaminternen Terminen wird das digitale Boards kaum verwendet. Stattdessen kommt ein analoges Board zum Einsatz, welches sich im Teambüro befindet. Es enthält sämtliche Stories des Teams, unabhängig vom Kunden, priorisiert. Im Daily dient dieses Board als Diskussiongrundlage und zeigt den aktuellen Stand der Arbeit. Daher muss es von den Teammitgliedern nach Erledigung einer Aufgabe stets auf dem Laufenden gehalten werden. Selbiges gilt für das digitale Board, damit auch der Kunde stets informiert ist. Eine genauere Erläuterung und Abgrenzung, sowie eine unternehmensweite Befragung zum Nutzen von analogen und digitalen Scrumboards sind in Kapitel \ref{sec:digitale_analoge_boards} zu finden. \\
\\
Die persönliche Motivation entsteht aus der eigenen Tätigkeit im Unternehmen kernpunkt und dem Willen den agilen Prozess praktisch und technisch zu verbessern. Die folgende Problemstellung entstand aus selbst beobachteten Problemen im Umgang mit digitalen und analogen Scrumboards. Dementsprechend bietet diese Arbeit auch die Möglichkeit die eigene Arbeitsweise, bzw. die Arbeitsweise im Team durch die eigene Forschung und Entwicklung zu verbessern.

\section{Problemstellung}
\label{sec:problemstellung}
Die parallele Nutzung von zwei verschiedenen Boards lässt Probleme entstehen. Die Boards müssen stets synchron gehalten werden, da sonst Inkonsistenzen entstehen. Das bedeutet, dass jedes Teammitglied nach Beendigung einer Aufgabe sowohl das analoge Board (durch Umhängen des Zettels), als auch das digitale Boards (durch Veränderung des Status) aktualisieren muss. Hierbei entsteht ein Mehraufwand, da die Story auf verschiedene Wege als erledigt markiert werden muss. Dieser Mehraufwand ist eine Quelle für individuelle Fehler, die zu Missverständnissen über den Stand der Arbeit führt. So kann ein nicht aktueller Stand im JIRA dafür sorgen, dass Kunden falsch informiert sind und auf Basis dessen Kommunikationsprobleme auftreten. Das Ausbessern dieser Fehler, sowie der entstehende Aufwand lassen sich meist nicht auf einen Kunden abrechnen, sodass hier ein finanzieller Schaden für das Unternehmen entstehen kann. \\
\\
Doch nicht nur Teammitglieder stoßen mit dieser Arbeitsweise auf Probleme. Zu Beginn eines Sprints bereitet bei kernpunkt der Scrum Master das Board vor. Das bedeutet, dass das Board geleert wird und neue Stories in Form von Zetteln ausgedruckt und ans Board gehangen werden. Darüber hinaus müssen Stories im JIRA einem Sprint zugewiesen werden und dieser gestartet werden. Da die Teams in einer Multiprojektumgebung arbeiten, muss dies für jeden Kunden einzeln geschehen. Auch hier muss die selbe Arbeit doppelt geleistet werden. Hinzu kommt, dass es während des Sprints zu Veränderungen an den Stories kommen kann. So verändert sich der Name einer Story oder neue Unteraufgaben, welche erst im laufenden Entwicklungsprozess klar werden, entstehen. Im JIRA können diese Änderungen direkt vorgenommen werden können, am analogen Board muss die Story hierzu neu ausgedruckt und aufgehangen werden. \\
\\
Diese zusätzlichen Arbeiten lenken das Team von ihrer eigentlichen Aufgabe ab. Außerhalb der finanziellen Aufwände ''steht diese Vorgehensweise in Konflikt mit dem Scrum Guide'' //REF: PP DOKU//. Zum Einen wird der Fokus des Teams eingeschränkt, obwohl dieser einer der zentralen Werte von Scrum ist \cite{agiledevelopmentwithscrum}. Außerdem widerspricht dies dem agilen Manifest, in dem ''Individuals and interactions over processes and tools'' gefordert wird \cite{agilemanifesto}. Das Board als Tool nimmt mehr Zeit in Anspruch und lenkt die Aufmerksamkeit weg von eigentlicher Arbeit und der Kommunikation innerhalb des Teams. Somit ist festzuhalten, dass das bisherige Verfahren nicht nur finanzielle und organisatorische Aufwände erzeugt, sondern auch der agilen Arbeitsweise nicht gerecht wird, nach welcher sich der Prozess eigentlich orientieren soll.  




