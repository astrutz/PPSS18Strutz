%!TEX root = ../main.tex
\chapter{Einleitung}
\label{cha:Einleitung}

Dieses Kapitel beschreibt die Domäne und das Problemfeld, in dem sich diese Arbeit bewegt. Außerdem wird die eigene Motivation, sowie die persönliche Bedeutung erläutert. Theoretische Grundlagen, insbesondere zur Agilität und Scrum, werden in diesem Kapitel vorrausgesetzt, jedoch erst in Kapitel \ref{cha:Grundlagen} erläutert.

\section{Kontext und Motivation der Arbeit}
\label{sec:kontext_motivation}
Das agile Framework Scrum hat in den letzten Jahren an Popularität gewonnen. So wird es laut der Studie ''Status Quo Agile 2016/2017'' der Hochschule Koblenz von 63\% der IT-Firmen -und Agenturen genutzt \cite{statusquoagile}. So auch im Unternehmen ''kernpunkt'', in dem //ich selbst tätig bin//. Dort wurde Scrum im November 2017 vorgestellt und das Unternehmen durch eine agile Transformation in die neue Arbeitsweise geführt. Grundlage stellt hier der ''Scrum Guide'' dar \cite{scrumguide}, der jedoch teilweise angepasst wurde, um mit dem Tagesgeschäft zu harmonieren. 
\\ \\
Während der Scrum Guide die Arbeit mit nur einem Kunden beschreibt, wurde der Prozess hier auf eine Multiprojektumgebung mit bis zu 8 verschiedenen Kunden und Projekten pro Team angepasst. Die verschiedenen Kunden müssen auf dem Teamboard getrennt werden, damit Stakeholder nur die für sie relevanten Stories sehen. Um dies umzusetzen, verwenden die verschiedenen Teams bei kernpunkt digitale Scrumboards, die auch für Kunden einsehbar sind. Auch wird durch das Board und die dort enthaltenen Stories die Kommunikation zwischen Kunde und Team bestimmt. In Dailies und weiteren teaminternen Terminen wird das digitale Boards kaum verwendet. Stattdessen kommt ein analoges Board zum Einsatz, welches sich im Teambüro befindet. Es enthält sämtliche Stories des Teams, unabhängig vom Kunden, priorisiert. Im Daily dient dieses Board als Diskussiongrundlage und zeigt den aktuellen Stand der Arbeit. Daher muss es von den Teammitgliedern nach Erledigung einer Aufgabe stets auf dem Laufenden gehalten werden. Selbiges gilt für das digitale Board, damit auch der Kunde stets informiert ist. Eine genauere Erläuterung und Abgrenzung, sowie eine unternehmensweite Befragung zum Nutzen von analogen und digitalen Scrumboards sind in Kapitel \ref{sec:digitale_analoge_boards} zu finden.
\\ \\
Die persönliche Motivation entsteht aus der eigenen Tätigkeit im Unternehmen und dem Willen den agilen Prozess praktisch und technisch zu verbessern.

\section{Problemstellung}
\label{sec:problemstellung}
In diesem Abschnitt werden die Tabellen aus dem vorigen Abschnitt im Text referenziert. 




