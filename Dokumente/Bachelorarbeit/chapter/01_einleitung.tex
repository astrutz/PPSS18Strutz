%!TEX root = ../main.tex
\chapter{Einleitung}
\label{cha:Einleitung}

In diesem Dokument werden einige Beispiele gegeben, die Ihnen das Erstellen Ihrer Arbeit erleichtern soll. Wir geben Ihnen hier Tipps zum Umgang mit \LaTeX, aber auch einige Hinweise zum Erstellen wissenschaftlicher Arbeiten. 

Wenn Sie ein neues Kapitel beginnen, beachten Sie bitte, dass dieses ebenfalls eine Einleitung aufweist. Das heißt, dass Kapitelüberschriften \textit{nicht} für sich alleine stehen und beispielsweise direkt die Überschrift des ersten Abschnittes folgt.


\section{Kontext und Motivation der Arbeit}
\label{sec:kontext_motivation}
Dieser Abschnitt gibt Beispiele für die Verwendung von Tabellen.

\section{Problemstellung}
\label{sec:problemstellung}
In diesem Abschnitt werden die Tabellen aus dem vorigen Abschnitt im Text referenziert. 




