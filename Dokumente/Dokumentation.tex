\documentclass[12pt,titlepage]{scrartcl}
\usepackage[ngerman]{babel}
\usepackage[utf8]{inputenc}
\usepackage{color}
\setkomafont{disposition}{\normalfont\bfseries}
\usepackage[a4paper,lmargin={3cm},rmargin={3cm},
tmargin={2.5cm},bmargin = {2.5cm}]{geometry}
\usepackage{graphicx}
\usepackage{tabularx}
\usepackage{amssymb}
\usepackage{amsthm}
\usepackage{caption}
\usepackage{float}
\usepackage{cite}
\usepackage{hyperref}

\begin{document}
	\begin{titlepage}
		\title{Dokumentation \\ \glqq Synchronisierung von analogen und digitalen Scrum-Boards\grqq{}} 
		\subtitle{Praxisprojekt}
		\author{Alexander Strutz \vspace{0.5cm}\\ Betreuer: 
		Prof. Dr. Matthias Böhmer,\\Oliver Blum}
 		\date{30. April 2019}
		\maketitle
	\end{titlepage}
	
	\tableofcontents
	
	\newpage
	
	\section{Domänenrecherche}
	Im Folgenden wird eine Einleitung in die Domäne, die Problemstellung mit ihren Folgen, sowie den verwendeten Paradigmen gegeben. Anschließend wird aus dieser Recherche eine Forschungsfrage abgeleitet.
		\subsection{Einleitung}
		Laut der Studie ''Status Quo Agile 2016/2017'' der Hochschule Koblenz verwenden 63\% der Software-Firmen und Agenturen agile Methoden in ihren täglichen Arbeitsprozessen \cite{hskob}. Agile, auch leichtgewichtig genannte Prozesse, können in verschiedenen Frameworks realisiert werden, jedoch orientieren sich alle an dem Rahmenwerk des ''Agilen Manifests''. Das agile Manifest setzt den Fokus auf die vier Werte ''Individuals and interactions over processes and tools'', ''Working software over comprehensive documentation'', ''Customer collaboration over contract negotiation'' und ''Responding to change over following a plan'' \cite{manifest}. Aus diesen Werten ergeben sich zwölf Prinzipien, die dem agilen Manifest angehören. Jedes agile Framework versucht die Werte und Prinzipien des Manifests umzusetzen und praktisch zu integrieren. \\
		Das in 82\% der befragten Firmen genutzte Framework ''Scrum'' gilt als meist genutzte agile Arbeitsmethode \cite{hskob}. Im Vordergrund stehen hierbei selbstorganisierte Teams. Diese Teams arbeiten interdisziplinär und bestehen aus drei bis neun Mitgliedern \cite{guide}. Sie ''entscheiden selbst, wie sie ihre Arbeit am besten erledigen, anstatt dieses durch andere Personen außerhalb des Teams vorgegeben zu bekommen'' \cite{guide}. Die zu erledigenden Aufgaben (''Stories'' genannt) werden durch den ''Product Owner'', einem Mitglied des Teams, in Absprache mit dem Kunden definiert und im ''Product Backlog'' priorisiert. Das Product Backlog listet alle Stories auf, die vom Product Owner erstellt wurden.\\
		Zu Beginn eines jeden Arbeitsinkrements, auch ''Sprint'' genannt, wählt das Team gemäß Priorisierung die Stories, die im folgenden Sprint zu bearbeiten sind. Jeder Sprint erhält ebenfalls ein Backlog, in dem Stories des Sprints nach ihrer Priorisierung gelistet enthalten sind. \\ 
		Die Stories werden anhand von ''Story Points'' in ihrer Komplexität eingeschätzt. Hierbei werden Punkte nicht nach Zeitaufwand, sondern nach vergleichbaren Stories geschätzt. Hierbei sind Zahlen der Fibonacci-Folge bis 20 möglich. Die Schätzung erfolgt gemeinsam durch das Team in einem sogenannten ''Refinement''. Der Sprint sollte so geplant werden, dass am Sprintende alle gewählten Stories erledigt sind. Eine Story kann hierbei drei Status einnehmen: ''To Do'', ''In Work'' oder ''Done''. Um das Team über die Status der einzelnen Stories informiert zu halten, gibt es täglich das ''Daily''. Hier berichtet jedes Teammitglied welche Story es am Vortag bearbeitet hat und ob sich der Status verändert hat. Um das Sprint Backlog zu visualisieren werden ''Boards'' verwendet. Bei diesen Boards kann es sich um Tafeln, Whiteboards oder ähnliches handeln, die die aktuellen Stories anzeigen. Hierbei gibt es drei Spalten, die die Status der Stories abbilden. Im Daily werden die Stories gemäß dem aktuellen Status verschoben. Die Organisation und Administration des Boards obliegt dem ''Scrum Master'', der ''dafür verantwortlich ist, Scrum entsprechend des Scrum Guides zu fördern	und zu unterstützen'' \cite{guide}. \\
		Das Board dient dem Team zur übersichtlichen Visualisierung des Sprints. Deshalb sollte es sich räumlich nah am Team befinden, bzw. gut vom Team erreichbar sein. Dem Kunden fehlt die Übersicht am Board jedoch, da dieser sich meistens nicht in direkter Nähe zum Team befindet. Daher wird die Darstellung der Stories oft durch Software wie ''Jira'' von der Firma Atlassian umgesetzt. Auch hier lassen sich Stories in einem Sprint verplanen, mit Punkten schätzen und mit einem Status versehen.
		
		\subsection{Problemstellung}
		Durch die Nutzung eines analogen Boards in der Nähe des Teams und eines digitalen Boards zur Sichtbarkeit beim Kunden treten Inkonsistenzen zwischen den Boards auf. Das Umhängen einer Story im Daily kann von einem System wie JIRA nicht erkannt werden. Somit muss der Status hier nachträglich verändert werden. Ebenso sieht es bei Änderungen im JIRA aus. Hierbei kann sich nicht nur der Status, sondern auch der Titel der Story, die Schätzung oder der Bearbeiter ändern. Dadurch muss die Story unter Umständen nicht nur verschoben, sondern durch den Scrum Master komplett neu ausgedruckt werden. \\
		Doch selbst wenn ein Board durch das Team stets synchron gehalten wird, können Probleme aufkommen. So werden Stories auf dem physischen Board nur im Daily verändert, während dies z.B. im JIRA ständig möglich ist. Daher muss man entweder im Daily das physische Board an das digitale Board angleichen oder das digitale auch erst im Daily aktualisieren. Die direkte Aktualisierung des digitalen Boards stellt jedoch einen Konflikt mit dem Scrum Guide dar, da dieser vorgibt, dass das Daily „ein entscheidendes Meeting zur Überprüfung und Anpassung“ ist \cite{guide}. Die zweite Lösung das digitale Board auch im Daily zu aktualisieren bringt neue Probleme mit sich. Zum einen muss sich ein Entwickler bei der Bearbeitung verschiedener Stories stets den aktuellen Status notieren, um im Daily nicht in Gefahr zu kommen etwas zu 			vergessen. Hierdurch entsteht ein Mehraufwand. Auch verringert so ein Vorgehen die Sichtbarkeit und Aktualität beim Kunden. \\
		Die ausschließliche Nutzung eines physischen Boards ist vom Kunden abhängig, da dies seine Anwesenheit im Daily voraussetzt, beispielsweise durch eine Videokonferenz. Dies ist jedoch nur möglich, wenn es sich nicht um eine Multiprojektumgebung handelt. In dem Falle wäre für jeden Kunden ein einzelnes Daily nötig, was den im Scrum Guide angegebenen Zeitaufwand von 15 Minuten übersteigen würde. Auch müsste das physische Boards so gestaltet sein, dass jeder Kunde nur den Teil sieht, der für ihn bestimmt ist, was sich kaum umsetzen lässt. \\
		Bei der ausschließlichen Nutzung eines digitalen Boards hingegen ist dies ohne weiteres möglich. So bietet Jira beispielweise die Möglichkeit verschiedene Projekte mit verschiedenen Sichtbarkeiten zu erstellen //REF
		Nur analog oder nur digital beschreiben -> Daraus entstehende Probleme

		\subsection{Folgen}
		\subsection{Paradigmen}
		\subsection{Stakeholder}
		\subsection{Forschungsfrage}
	\newpage	
	\section{Zielsetzung}
		\subsection{Strategische Ziele}
		\subsection{Taktische Ziele}
		\subsection{Operative Ziele}
	\newpage	
	\section{Marktrecherche}
		\subsection{Produkt 1}
		\subsection{Produkt 2}
		\subsection{Fazit}
	\newpage
	\section{Alleinstellungsmerkmale}	
	\newpage
	\section{Proof of Concept}
	Im Folgenden werden Proofs of Concept aufgelistet, die die Kernfunktionen des Systems abbilden und mögliche Fehler abfangen.

\begin{table}[H]
\centering
\caption{\textbf{POC 1: Story einer Karte zuweisen}}
\label{poc1}
\begin{tabularx}{\linewidth}{|r|X|}
\hline
Beschreibung  & Über das JIRA lassen sich Stories zu Karten zuordnen, die dann Informationen zur Story beinhalten. qwat                                  \\ \hline
Exitkriterium & Es lässt sich aus einer Liste von verfügbaren Karten eine auswählen, die Informationen zur Story werden versendet. \\ \hline
Failkriterium & Es werden keine Karten angezeigt, obwohl es verfügbare Karten im Netz gibt. \\ \hline
Fallback      & Es gibt die Möglichkeit Stories der ''nächsten verfügbaren Karte'' zuzuweisen. Sobald sich eine Karte wieder im Netz befindet erhält sich automatisch die Informationen der Story. \\ \hline
\end{tabularx}
\end{table}

\begin{table}[H]
\centering
\caption{\textbf{POC 2: Story auf einer Karte anzeigen}}
\label{poc2}
\begin{tabularx}{\linewidth}{|r|X|}
\hline
Beschreibung  & Eine Karte kann durch ihr Display Informationen wie Titel und Kürzel einer Story anzeigen. \\ \hline
Exitkriterium & Die Karte enthält die aktuellen Informationen zur Story. \\ \hline
Failkriterium & Die Karte enthält Informationen zur Story, die veraltet sind, da die Story im JIRA verändert wurde.
 \\ \hline
Fallback      & Der Nutzer hat die Möglichkeit eine Aktualisierung der Karte anzustoßen. Außerdem kann sich die Karte in einem bestimmten Intervall selbst aktualisieren.
 \\ \hline
\end{tabularx}
\end{table}

\begin{table}[H]
\centering
\caption{\textbf{POC 3: Story einem Teammitglied zuweisen}}
\label{poc3}
\begin{tabularx}{\linewidth}{|r|X|}
\hline
Beschreibung  & Durch Anlegen eines RFID-Tags kann sich ein Teammitglied eine Story selbst zuweisen.  \\ \hline
Exitkriterium & Sowohl im JIRA als auch auf der Karte ist die Story dem Besitzer des RFID-Tags zugewiesen. \\ \hline
Failkriterium & Das System kann dem RFID-Tag keinen Nutzer zuordnen.
 \\ \hline
Fallback      & Wenn ein Nutzer im JIRA einer Story zugewiesen wird, wird die Karte wie in PoC 2 aktualisiert. 
 \\ \hline
\end{tabularx}
\end{table}

\begin{table}[H]
\centering
\caption{\textbf{POC 4: Status der Story verändern}}
\label{poc4}
\begin{tabularx}{\linewidth}{|r|X|}
\hline
Beschreibung  & Durch Verschieben einer Karte in eine andere Spalte wird der Status der Story angepasst.  \\ \hline
Exitkriterium & Nach Verschiebung hat die Story im JIRA einen anderen Status. \\ \hline
Failkriterium & Der Status der Story wurde im JIRA und nicht auf dem Board verändert. \\ \hline
Fallback      & Wenn eine Karte in einer anderen Spalte hängt als im JIRA wird dies auf der Karte angezeigt, bis sie verschoben wurde.  \\ \hline
\end{tabularx}
\end{table}
	\newpage	
	\section{Technologieentscheidung}
		\subsection{Mikrocontroller}
		\subsection{Display}
		\subsection{RFID-Reader}
		\subsection{Sensorik am Board}
		\subsection{Server}
		\subsection{Schnittstellen}
	\newpage	
	\section{Architektur}
	\newpage
	\section{Implementation}
	\newpage	
	\section{Fazit}
	\newpage

	
\bibliography{Dokumentation}
 	\bibliographystyle{unsrt}
\end{document}
